% Options for packages loaded elsewhere
\PassOptionsToPackage{unicode}{hyperref}
\PassOptionsToPackage{hyphens}{url}
\PassOptionsToPackage{dvipsnames,svgnames,x11names}{xcolor}
%
\documentclass[
  letterpaper,
  DIV=11,
  numbers=noendperiod]{scrreprt}

\usepackage{amsmath,amssymb}
\usepackage{iftex}
\ifPDFTeX
  \usepackage[T1]{fontenc}
  \usepackage[utf8]{inputenc}
  \usepackage{textcomp} % provide euro and other symbols
\else % if luatex or xetex
  \usepackage{unicode-math}
  \defaultfontfeatures{Scale=MatchLowercase}
  \defaultfontfeatures[\rmfamily]{Ligatures=TeX,Scale=1}
\fi
\usepackage{lmodern}
\ifPDFTeX\else  
    % xetex/luatex font selection
\fi
% Use upquote if available, for straight quotes in verbatim environments
\IfFileExists{upquote.sty}{\usepackage{upquote}}{}
\IfFileExists{microtype.sty}{% use microtype if available
  \usepackage[]{microtype}
  \UseMicrotypeSet[protrusion]{basicmath} % disable protrusion for tt fonts
}{}
\makeatletter
\@ifundefined{KOMAClassName}{% if non-KOMA class
  \IfFileExists{parskip.sty}{%
    \usepackage{parskip}
  }{% else
    \setlength{\parindent}{0pt}
    \setlength{\parskip}{6pt plus 2pt minus 1pt}}
}{% if KOMA class
  \KOMAoptions{parskip=half}}
\makeatother
\usepackage{xcolor}
\usepackage[landscape]{geometry}
\setlength{\emergencystretch}{3em} % prevent overfull lines
\setcounter{secnumdepth}{-\maxdimen} % remove section numbering
% Make \paragraph and \subparagraph free-standing
\makeatletter
\ifx\paragraph\undefined\else
  \let\oldparagraph\paragraph
  \renewcommand{\paragraph}{
    \@ifstar
      \xxxParagraphStar
      \xxxParagraphNoStar
  }
  \newcommand{\xxxParagraphStar}[1]{\oldparagraph*{#1}\mbox{}}
  \newcommand{\xxxParagraphNoStar}[1]{\oldparagraph{#1}\mbox{}}
\fi
\ifx\subparagraph\undefined\else
  \let\oldsubparagraph\subparagraph
  \renewcommand{\subparagraph}{
    \@ifstar
      \xxxSubParagraphStar
      \xxxSubParagraphNoStar
  }
  \newcommand{\xxxSubParagraphStar}[1]{\oldsubparagraph*{#1}\mbox{}}
  \newcommand{\xxxSubParagraphNoStar}[1]{\oldsubparagraph{#1}\mbox{}}
\fi
\makeatother


\providecommand{\tightlist}{%
  \setlength{\itemsep}{0pt}\setlength{\parskip}{0pt}}\usepackage{longtable,booktabs,array}
\usepackage{calc} % for calculating minipage widths
% Correct order of tables after \paragraph or \subparagraph
\usepackage{etoolbox}
\makeatletter
\patchcmd\longtable{\par}{\if@noskipsec\mbox{}\fi\par}{}{}
\makeatother
% Allow footnotes in longtable head/foot
\IfFileExists{footnotehyper.sty}{\usepackage{footnotehyper}}{\usepackage{footnote}}
\makesavenoteenv{longtable}
\usepackage{graphicx}
\makeatletter
\def\maxwidth{\ifdim\Gin@nat@width>\linewidth\linewidth\else\Gin@nat@width\fi}
\def\maxheight{\ifdim\Gin@nat@height>\textheight\textheight\else\Gin@nat@height\fi}
\makeatother
% Scale images if necessary, so that they will not overflow the page
% margins by default, and it is still possible to overwrite the defaults
% using explicit options in \includegraphics[width, height, ...]{}
\setkeys{Gin}{width=\maxwidth,height=\maxheight,keepaspectratio}
% Set default figure placement to htbp
\makeatletter
\def\fps@figure{htbp}
\makeatother

\KOMAoption{captions}{tableheading}
\makeatletter
\@ifpackageloaded{caption}{}{\usepackage{caption}}
\AtBeginDocument{%
\ifdefined\contentsname
  \renewcommand*\contentsname{Table of contents}
\else
  \newcommand\contentsname{Table of contents}
\fi
\ifdefined\listfigurename
  \renewcommand*\listfigurename{List of Figures}
\else
  \newcommand\listfigurename{List of Figures}
\fi
\ifdefined\listtablename
  \renewcommand*\listtablename{List of Tables}
\else
  \newcommand\listtablename{List of Tables}
\fi
\ifdefined\figurename
  \renewcommand*\figurename{Figure}
\else
  \newcommand\figurename{Figure}
\fi
\ifdefined\tablename
  \renewcommand*\tablename{Table}
\else
  \newcommand\tablename{Table}
\fi
}
\@ifpackageloaded{float}{}{\usepackage{float}}
\floatstyle{ruled}
\@ifundefined{c@chapter}{\newfloat{codelisting}{h}{lop}}{\newfloat{codelisting}{h}{lop}[chapter]}
\floatname{codelisting}{Listing}
\newcommand*\listoflistings{\listof{codelisting}{List of Listings}}
\makeatother
\makeatletter
\makeatother
\makeatletter
\@ifpackageloaded{caption}{}{\usepackage{caption}}
\@ifpackageloaded{subcaption}{}{\usepackage{subcaption}}
\makeatother

\ifLuaTeX
  \usepackage{selnolig}  % disable illegal ligatures
\fi
\usepackage{bookmark}

\IfFileExists{xurl.sty}{\usepackage{xurl}}{} % add URL line breaks if available
\urlstyle{same} % disable monospaced font for URLs
\hypersetup{
  pdftitle={Value Chains},
  pdfauthor={Greg Alward},
  colorlinks=true,
  linkcolor={blue},
  filecolor={Maroon},
  citecolor={Blue},
  urlcolor={Blue},
  pdfcreator={LaTeX via pandoc}}


\title{Value Chains}
\author{Greg Alward}
\date{2024-09-25}

\begin{document}
\maketitle


\chapter{Purpose and Objective}\label{purpose-and-objective}

Resource-Product Value chain analysis and visualization using I-O
accounts as a data source.

A premise of this project is that double-entry I-O accounts contain a
sufficient data set to derive directed acyclical graphs that can in turn
be visualized. The graph (composed of nodes and edges) depicts the order
and sequence of activities in a process that transforms a primary
resource (e.g.~wood from standing trees) into an end product (e.g.,
electricity generated by burning wood - biomass power generation).
Graphs of this type are commonly referred to as ``supply chains'' or
``resource value chains'' or ``resource-product value chains (RPVC)''.

This project describes and illustrates methods used extract and
visualize RPVCs from I-O accounts.

\section{1. National Income and Product
Accounts}\label{national-income-and-product-accounts}

Input-Output accounts are an artifact in a System of National Accounts
or SNA (also referred to as National Income and Product Accounts or
NIPA). Fundamental accounting identities hold for NIPA accounts and I-O
accounts inherit the same identities.

National Income and Product identities (in scalar form): \[
\begin{align}
  v = c+i+g+(e-m)   && \text{(National Income = National Product)}\\
  m+v = c+i+g+e     && \text{(GNI + imports = GNP)}\\
  m+v = f     && \text{(imports + GNI = Total Final Demand = f)}\\
\end{align}
\]

I-O accounts add an intermediate output and outlay variables to the
identity to introduce the concept of total industry output and outlay:

\[
\begin{align}
  x+m+v = x+f     && \text{(x=intermediate output/outlay; (x+f) and (m+v)= total output)}\\
\end{align}
\]

\section{2. Core Form I-O Accounts}\label{core-form-i-o-accounts}

\subsection{2.1 Core Form I-O Accounts}\label{core-form-i-o-accounts-1}

The fundamental accounting identities (i.e., double-entry identities)
for I-O are shown in equations (1) and (2) below. These matrix equations
describe the \emph{Core Form} (vs the \emph{Leontief Form}) of I-O
accounts

\[
\begin{align}
    X + f = t && \text{(Intermediate Demand + Final Demand = Total Output) (1)}\\
    X + V + M = q && \text{(Intermediate Outlay + Value Added + Imports = Total Outlay) (2)}\\
\end{align}
\]

\subsection{2,2 Direct input
coefficients}\label{direct-input-coefficients}

Using the \emph{Core Form} equations (1) and (2), and assuming a
multi-sector set of accounts, we cannot associate the multi-sector
pattern of inputs (V and M) with the multi-sector pattern of final
products f.~In order to do this, we need algebraically transform the
\emph{Core Form} accounts into the \emph{Leontief Form} of the same I-O
accounts. To accomplish the algebraic transformation, start with the
assumption that the pattern of inputs for an activity's product can be
given by a vector of \emph{direct coefficients} as shown in equations
(3) - (5).

\[
\begin{align}
    \hat{q}^{-1}X = _{i}A && \text{(Inter-industry Direct Input Coefficients matrix iA) (3)}\\
    \hat{q}^{-1}V = _{v}A && \text{(Value Added Direct Input Coefficients matrix vA) (4)}\\
    \hat{q}^{-1}M = _{m}A && \text{(Import Direct Input Coefficients matrix mA) (5)}\\
\end{align}
\]

Total Industry Outlay Vector \([q]_{1\times 6}\):

\[\mathbf{[q]_{1\times 6}} = \left[\begin{array}
{rrr}
q_{1} \\
q_{2} \\
q_{3} \\
q_{4} \\
q_{5} \\
q_{6} \\
\end{array}\right]
\]

Total Outlay Inverse (Reciprocals) Matrix
\([\hat{q}^{-1}]_{6\times 6}\):

\[\mathbf {[\hat{q}^{-1}]_{6\times 6}} = \left[\begin{array}
{rrr}
(\frac{1} q_{1}) & 0 & 0 & 0 & 0 & 0 \\
0 & (\frac{1} q_{2}) & 0 & 0 & 0 & 0 \\
0 & 0 & (\frac{1} q_{3}) & 0 & 0 & 0 \\
0 & 0 & 0 & (\frac{1} q_{4}) & 0 & 0 \\
0 & 0 & 0 & 0 & (\frac{1} q_{5}) & 0 \\
0 & 0 & 0 & 0 & 0 & (\frac{1} q_{6}) \\
\end{array}\right]
\]

\subsubsection{2.2.1 Inter-industry Direct
Coefficients}\label{inter-industry-direct-coefficients}

\[
\begin{align}
    \hat{q}^{-1}X = _{i}A && \text{(Inter-industry Direct Input Coefficients matrix iA) (3)}\\
\end{align}
\]

Inter-industry Transactions Matrix \([X]_{6\times 6}\):

\[\mathbf{[X]_{6\times 6}} = \left[\begin{array}
{rrr}
x_{11} & x_{12} & x_{13} & x_{14} & x_{15} & x_{16} \\
x_{21} & x_{22} & x_{23} & x_{24} & x_{25} & x_{26} \\
x_{31} & x_{32} & x_{33} & x_{34} & x_{35} & x_{36} \\
x_{41} & x_{42} & x_{43} & x_{44} & x_{45} & x_{46} \\
x_{51} & x_{52} & x_{53} & x_{54} & x_{55} & x_{56} \\
x_{61} & x_{62} & x_{63} & x_{64} & x_{65} & x_{66} \\
\end{array}\right]
\]

Inter-Industry Explicit Computation of Direct Coefficients Matrix
\([\hat{q}^{-1}X]_{6\times 6}=[_{i}A]_{6\times 6}\):

\[\mathbf{[\hat{q}^{-1}X]_{6\times 6}=[_{i}A]_{6\times 6}} = \left[\begin{array}
{rrr}
((\frac{1} q_{1})*x_{11})=_{i}a_{11} & ((\frac{1} q_{1})*x_{12})=_{i}a_{12} & ((\frac{1} q_{1})*x_{13})=_{i}a_{13} & ((\frac{1} q_{1})*x_{13})=_{i}a_{14} & ((\frac{1} q_{1})*x_{15})=_{i}a_{15} & ((\frac{1} q_{1})*x_{16})=a_{16} \\
((\frac{1} q_{2})*x_{21})=_{i}a_{21} & ((\frac{1} q_{2})*x_{22})=_{i}a_{22} & ((\frac{1} q_{2})*x_{23})=_{i}a_{23} & ((\frac{1} q_{2})*x_{23})=_{i}a_{24} & ((\frac{1} q_{2})*x_{25})=_{i}a_{25} & ((\frac{1} q_{2})*x_{26})=a_{26} \\
((\frac{1} q_{3})*x_{31})=_{i}a_{31} & ((\frac{1} q_{3})*x_{32})=_{i}a_{32} & ((\frac{1} q_{3})*x_{33})=_{i}a_{33} & ((\frac{1} q_{3})*x_{33})=_{i}a_{34} & ((\frac{1} q_{3})*x_{35})=_{i}a_{35} & ((\frac{1} q_{3})*x_{36})=a_{36} \\
((\frac{1} q_{4})*x_{41})=_{i}a_{41} & ((\frac{1} q_{4})*x_{42})=_{i}a_{42} & ((\frac{1} q_{4})*x_{43})=_{i}a_{43} & ((\frac{1} q_{4})*x_{43})=_{i}a_{44} & ((\frac{1} q_{4})*x_{45})=_{i}a_{45} & ((\frac{1} q_{4})*x_{46})=a_{46} \\
((\frac{1} q_{5})*x_{51})=_{i}a_{51} & ((\frac{1} q_{5})*x_{52})=_{i}a_{52} & ((\frac{1} q_{5})*x_{53})=_{i}a_{53} & ((\frac{1} q_{5})*x_{53})=_{i}a_{54} & ((\frac{1} q_{5})*x_{55})=_{i}a_{55} & ((\frac{1} q_{5})*x_{56})=a_{56} \\
((\frac{1} q_{6})*x_{61})=_{i}a_{61} & ((\frac{1} q_{6})*x_{62})=_{i}a_{62} & ((\frac{1} q_{6})*x_{63})=_{i}a_{63} & ((\frac{1} q_{6})*x_{63})=_{i}a_{64} & ((\frac{1} q_{6})*x_{65})=_{i}a_{65} & ((\frac{1} q_{6})*x_{66})=a_{66} \\
\end{array}\right]
\]

Inter-Industry Direct Coefficients Matrix \([_{i}A]_{6\times 6}\):

\[\mathbf{[_{i}A]_{6\times 6}} = \left[\begin{array}
{rrr}
_{i}a_{11} & _{i}a_{12} & _{i}a_{13} & _{i}a_{14} & _{i}a_{15} & a_{16} \\
_{i}a_{21} & _{i}a_{22} & _{i}a_{23} & _{i}a_{24} & _{i}a_{25} & a_{26} \\
_{i}a_{31} & _{i}a_{32} & _{i}a_{33} & _{i}a_{34} & _{i}a_{35} & a_{36} \\
_{i}a_{41} & _{i}a_{42} & _{i}a_{43} & _{i}a_{44} & _{i}a_{45} & a_{46} \\
_{i}a_{51} & _{i}a_{52} & _{i}a_{53} & _{i}a_{54} & _{i}a_{55} & a_{56} \\
_{i}a_{61} & _{i}a_{62} & _{i}a_{63} & _{i}a_{64} & _{i}a_{65} & a_{66} \\
\end{array}\right]
\]

\subsubsection{2.2.2 Total Value Added Direct
Coefficients}\label{total-value-added-direct-coefficients}

\[
\begin{align}
    \hat{q}^{-1}V = _{v}A && \text{(Value Added Direct Input Coefficients matrix vA) (4)}\\
\end{align}
\]

Total Value Added Matrix \([V]_{6\times 6}\):

\[\mathbf {[V]_{6\times 6}} = \left[\begin{array}
{rrr}
v_{11} & 0 & 0 & 0 & 0 & 0 \\
0 & v_{22} & 0 & 0 & 0 & 0 \\
0 & 0 & v_{33} & 0 & 0 & 0 \\
0 & 0 & 0 & v_{44} & 0 & 0 \\
0 & 0 & 0 & 0 & v_{55} & 0 \\
0 & 0 & 0 & 0 & 0 & v_{66} \\
\end{array}\right]
\]

Explicit Computation of Total Value Added Direct Coefficients Matrix
\([\hat{q}^{-1}v]_{6\times 6}=[_{v}A]_{6\times 6}\):

\[\mathbf{[\hat{q}^{-1}V]_{6\times 6}=[_{v}A]_{6\times 6}} = \left[\begin{array}
{rrr}
((\frac{1} q_{1})*v_{11})=_{v}a_{11} & 0 & 0 & 0 & 0 & 0 \\
0 & ((\frac{1} q_{2})*v_{22})=_{v}a_{22} & 0 & 0 & 0 & 0 \\
0 & 0 & ((\frac{1} q_{3})*v_{33})=_{v}a_{33} & 0 & 0 & 0 \\
0 & 0 & 0 & ((\frac{1} q_{4})*v_{43})=_{v}a_{44} & 0 & 0 \\
0 & 0 & 0 & 0 & ((\frac{1} q_{5})*v_{55})=_{v}a_{55} & 0 \\
0 & 0 & 0 & 0 & 0 & ((\frac{1} q_{6})*v_{66})=_{v}a_{66} \\
\end{array}\right]
\]

Total Value Added Direct Coefficients Matrix \([_{v}A]_{6\times 6}\):

\[\mathbf{[_{v}A]_{6\times 6}} = \left[\begin{array}
{rrr}
_{v}a_{11} &0 & 0 & 0 & 0 & 0 \\
0 & _{v}a_{22} & 0 & 0 & 0 & 0 \\
0 & 0 & _{v}a_{33} & 0 & 0 & 0 \\
0 & 0 & 0 & _{v}a_{44} & 0 & 0 \\
0 & 0 & 0 & 0 & _{v}a_{55} & 0 \\
0 & 0 & 0 & 0 & 0 & _{v}a_{66} \\
\end{array}\right]
\]

\subsubsection{2.2.3 Total Imports Direct
Coefficients}\label{total-imports-direct-coefficients}

\[
\begin{align}
    \hat{q}^{-1}M = _{m}A && \text{(Import Direct Input Coefficients matrix mA) (5)}\\
\end{align}
\]

Total Imports Matrix \([M]_{6\times 6}\):

\[\mathbf {[M]_{6\times 6}} = \left[\begin{array}
{rrr}
m_{11} & 0 & 0 & 0 & 0 & 0 \\
0 & m_{22} & 0 & 0 & 0 & 0 \\
0 & 0 & m_{33} & 0 & 0 & 0 \\
0 & 0 & 0 & m_{44} & 0 & 0 \\
0 & 0 & 0 & 0 & m_{55} & 0 \\
0 & 0 & 0 & 0 & 0 & m_{66} \\
\end{array}\right]
\]

Explicit Computation of Total Imports Direct Coefficients Matrix
\([\hat{q}^{-1}M]_{6\times 6}=[_{m}A]_{6\times 6}\):

\[\mathbf{[\hat{q}^{-1}M]_{6\times 6}=[_{m}A]_{6\times 6}} = \left[\begin{array}
{rrr}
((\frac{1} q_{1})*m_{11})=_{m}a_{11} & 0 & 0 & 0 & 0 & 0 \\
0 & ((\frac{1} q_{2})*m_{22})=_{m}a_{22} & 0 & 0 & 0 & 0 \\
0 & 0 & ((\frac{1} q_{3})*m_{33})=_{m}a_{33} & 0 & 0 & 0 \\
0 & 0 & 0 & ((\frac{1} q_{4})*m_{43})=_{m}a_{44} & 0 & 0 \\
0 & 0 & 0 & 0 & ((\frac{1} q_{5})*m_{55})=_{m}a_{55} & 0 \\
0 & 0 & 0 & 0 & 0 & ((\frac{1} q_{6})*m_{66})=_{m}a_{66} \\
\end{array}\right]
\]

Total Value Added Direct Coefficients Matrix \([_{m}A]_{6\times 6}\):

\[\mathbf{[_{m}A]_{6\times 6}} = \left[\begin{array}
{rrr}
_{m}a_{11} &0 & 0 & 0 & 0 & 0 \\
0 & _{m}a_{22} & 0 & 0 & 0 & 0 \\
0 & 0 & _{m}a_{33} & 0 & 0 & 0 \\
0 & 0 & 0 & _{m}a_{44} & 0 & 0 \\
0 & 0 & 0 & 0 & _{m}a_{55} & 0 \\
0 & 0 & 0 & 0 & 0 & _{m}a_{66} \\
\end{array}\right]
\]

Direct Coefficients Matrix
\([\hat{q}^{-1}X]_{6\times 6}=[_{i}A]_{6\times 6}\):

\[\mathbf{[\hat{q}^{-1}X]_{6\times 6}=[_{i}A]_{6\times 6}} = \left[\begin{array}
{rrr}
((\frac{1} q_{1})*x_{11})=_{i}a_{11} & ((\frac{1} q_{1})*x_{12})=_{i}a_{12} & ((\frac{1} q_{1})*x_{13})=_{i}a_{13} & ((\frac{1} q_{1})*x_{13})=_{i}a_{14} & ((\frac{1} q_{1})*x_{15})=_{i}a_{15} & ((\frac{1} q_{1})*x_{16})=a_{16} \\
((\frac{1} q_{2})*x_{21})=_{i}a_{21} & ((\frac{1} q_{2})*x_{22})=_{i}a_{22} & ((\frac{1} q_{2})*x_{23})=_{i}a_{23} & ((\frac{1} q_{2})*x_{23})=_{i}a_{24} & ((\frac{1} q_{2})*x_{25})=_{i}a_{25} & ((\frac{1} q_{2})*x_{26})=a_{26} \\
((\frac{1} q_{3})*x_{31})=_{i}a_{31} & ((\frac{1} q_{3})*x_{32})=_{i}a_{32} & ((\frac{1} q_{3})*x_{33})=_{i}a_{33} & ((\frac{1} q_{3})*x_{33})=_{i}a_{34} & ((\frac{1} q_{3})*x_{35})=_{i}a_{35} & ((\frac{1} q_{3})*x_{36})=a_{36} \\
((\frac{1} q_{4})*x_{41})=_{i}a_{41} & ((\frac{1} q_{4})*x_{42})=_{i}a_{42} & ((\frac{1} q_{4})*x_{43})=_{i}a_{43} & ((\frac{1} q_{4})*x_{43})=_{i}a_{44} & ((\frac{1} q_{4})*x_{45})=_{i}a_{45} & ((\frac{1} q_{4})*x_{46})=a_{46} \\
((\frac{1} q_{5})*x_{51})=_{i}a_{51} & ((\frac{1} q_{5})*x_{52})=_{i}a_{52} & ((\frac{1} q_{5})*x_{53})=_{i}a_{53} & ((\frac{1} q_{5})*x_{53})=_{i}a_{54} & ((\frac{1} q_{5})*x_{55})=_{i}a_{55} & ((\frac{1} q_{5})*x_{56})=a_{56} \\
((\frac{1} q_{6})*x_{61})=_{i}a_{61} & ((\frac{1} q_{6})*x_{62})=_{i}a_{62} & ((\frac{1} q_{6})*x_{63})=_{i}a_{63} & ((\frac{1} q_{6})*x_{63})=_{i}a_{64} & ((\frac{1} q_{6})*x_{65})=_{i}a_{65} & ((\frac{1} q_{6})*x_{66})=a_{66} \\
\end{array}\right]
\]

\section{3. Inter-Industry Adjacency
Matrix}\label{inter-industry-adjacency-matrix}

\subsection{3.1 Identifying an RPVC Path in an Adjacency
Matrix}\label{identifying-an-rpvc-path-in-an-adjacency-matrix}

The \([_{i}A]\) matrix is commonly referred to as the \emph{Direct
Coefficients} matrix. The term \emph{Direct} refers to the
\emph{one-step} relationships between an industry activity and its input
suppliers. That is, each industry activity is one transaction (i.e.,
\emph{one step}) removed from each of its suppliers. In aggregate, each
industry activity and its suppliers have a \emph{one-to-many}
relationship (i.e., one activity has many suppliers). Each activity's
column vector in the \([_{i}A]\) matrix describes that activity's
\emph{one-to-many} relationships with its input suppliers.

An \emph{Adjacency Matrix} derived from the \([_{i}A]\) matrix can be
used to determine the number of one-step relationships (i.e.,
transactions) between any pair of purchasing and suppling activities. In
other words, an \emph{Adjacency Matrix} can reveal the path and distance
(number of steps) between an origin activity and a destination activity
(i.e., an RPVC).

Derive adjacency matrix \([B]\) from the \([_{i}A]\) matrix. Successive
powers of \([B^{n}]\) will identify successive paths of step-length
\textbf{\emph{n}}.

To illustrate an adjacency matrix, consider a portion of an \([_{i}A]\)
matrix with only wood-related activities. Assume industry
\textbf{\emph{1}} is \emph{Grow}, industry \textbf{\emph{2}} is
\emph{Harvest}, industry \textbf{\emph{3}} is \emph{Saw Mill}, industry
\textbf{\emph{4}} is \emph{Residuals}, industry \textbf{\emph{5}} is
\emph{Bio-Power}, and industry \textbf{\emph{6}} is \emph{Other}. Assume
the commodity produced by industry \textbf{\emph{1}} is \emph{Stumpage}
(\emph{m\textsuperscript{3}} of solid wood), the commodity produced by
industry \textbf{\emph{2}} is \emph{Logs} (containing
\emph{m\textsuperscript{3}} of solid wood), the commodity produced by
industry \textbf{\emph{3}} is \emph{Dimension Lumber} (containing
\emph{m\textsuperscript{3}} of solid wood), the commodity produced by
industry \textbf{\emph{4}} is \emph{Sawdust Residuals} (containing
\emph{m\textsuperscript{3}} of solid wood), the commodity produced by
industry \textbf{\emph{5}} is \emph{Electricity} (generated by using
\emph{m\textsuperscript{3}} of solid wood \emph{Sawdust Residuals} as
fuel), and commodity produced by industry \textbf{\emph{6}} is
\emph{Other}.

The Inter-industry Direct Coefficients Matrix \([_{i}A]_{6\times 6}\):

\[\mathbf{[_{i}A]_{6\times 6}} = \left[\begin{array}
{rrr}
_{i}a_{11} & _{i}a_{12} & _{i}a_{13} & _{i}a_{14} & _{i}a_{15} & _{i}a_{16} \\
_{i}a_{21} & _{i}a_{22} & _{i}a_{23} & _{i}a_{24} & _{i}a_{25} & _{i}a_{26} \\
_{i}a_{31} & _{i}a_{32} & _{i}a_{33} & _{i}a_{34} & _{i}a_{35} & _{i}a_{36} \\
_{i}a_{41} & _{i}a_{42} & _{i}a_{43} & _{i}a_{44} & _{i}a_{45} & _{i}a_{46} \\
_{i}a_{51} & _{i}a_{52} & _{i}a_{53} & _{i}a_{54} & _{i}a_{55} & _{i}a_{56} \\
_{i}a_{61} & _{i}a_{62} & _{i}a_{63} & _{i}a_{64} & _{i}a_{65} & _{i}a_{66} \\
\end{array}\right]
\]

Inter-industry Direct Coefficients Matrix \([_{i}A]_{6\times 6}\) where
\(_{i}a_{nn}\neq{0}\):

\[\mathbf{[_{i}A]_{6\times 6}} = \left[\begin{array}
{rrr}
_{i}a_{11} & _{i}a_{12} & 0 & 0 & 0 & 0 \\
0 & _{i}a_{22} & _{i}a_{23} & 0 & 0 & 0 \\
0 & 0 & _{i}a_{33} & _{i}a_{34} & 0 & 0 \\
0 & 0 & 0 & _{i}a_{44} & _{i}a_{45} & 0 \\
0 & 0 & 0 & 0 & 0 & 0 \\
0 & 0 & 0 & 0 & 0 & 0 \\
\end{array}\right]
\]

Adjacency matrix \([B^{1}_{6\times 6}]\) (no 1-step path between
Activity 5 and Activity 1 indicated by \(b^{1}_{15}=0\)):

\[\mathbf{[B^{1}]_{6\times 6}} = \left[\begin{array}
{rrr}
1 & 1 & 0 & 0 & 0 & 0 \\
0 & 1 & 1 & 0 & 0 & 0 \\
0 & 0 & 1 & 1 & 0 & 0 \\
0 & 0 & 0 & 1 & 1 & 0 \\
0 & 0 & 0 & 0 & 0 & 0 \\
0 & 0 & 0 & 0 & 0 & 0 \\
\end{array}\right]
\]

Adjacency Matrix \([B^{2}_{6\times 6}]\) (no 2-step path between
Activity 5 and Activity 1 indicated by \(b^{2}_{15}=0\)):

\[\mathbf{[B^{2}]_{6\times 6}} = \left[\begin{array}
{rrr}
1 & 2 & 1 & 0 & 0 & 0 \\
0 & 1 & 2 & 1 & 0 & 0 \\
0 & 0 & 1 & 2 & 1 & 0 \\
0 & 0 & 0 & 1 & 1 & 0 \\
0 & 0 & 0 & 0 & 0 & 0 \\
0 & 0 & 0 & 0 & 0 & 0 \\
\end{array}\right]
\]

Adjacency Matrix \([B^{3}_{6\times 6}]\) (no 3-step path between
Activity 5 and Activity 1 indicated by \(b^{3}_{15}=0\)):

\[\mathbf{[B^{3}]_{6\times 6}} = \left[\begin{array}
{rrr}
1 & 3 & 3 & 1 & 0 & 0 \\
0 & 1 & 3 & 3 & 1 & 0 \\
0 & 0 & 1 & 3 & 2 & 0 \\
0 & 0 & 0 & 1 & 1 & 0 \\
0 & 0 & 0 & 0 & 0 & 0 \\
0 & 0 & 0 & 0 & 0 & 0 \\
\end{array}\right]
\]

Adjacency Matrix \([B^{4}_{6\times 6}]\) (one 4-step path between
Activity 5 and Activity 1 indicated by \(b^{4}_{15}=1\)):

\[\mathbf{[B^{4}]_{6\times 6}} = \left[\begin{array}
{rrr}
1 & 4 & 6 & 4 & 1 & 0 \\
0 & 1 & 4 & 5 & 2 & 0 \\
0 & 0 & 1 & 2 & 1 & 0 \\
0 & 0 & 0 & 1 & 1 & 0 \\
0 & 0 & 0 & 0 & 0 & 0 \\
0 & 0 & 0 & 0 & 0 & 0 \\
\end{array}\right]
\]

The \emph{trace} of the \emph{path} of \emph{length} 4 from Activity 5
to Activity 1 in \([B^{4}]\):

\textbf{step 1: (b\textsubscript{55} \textgreater{}
b\textsubscript{45}), step 2: (b\textsubscript{43} \textgreater{}
b\textsubscript{33}), step 3: (b\textsubscript{32} \textgreater{}
b\textsubscript{22}), step 4: (b\textsubscript{21} \textgreater{}
b\textsubscript{11})}

A \emph{path} like this can be used below to determine the order and
number of steps (column-space expansions) necessary to decompose a
specific RPVC from a \emph{Leontief Form} product vector.

\subsection{3.2 Visual Analysis of a Path in Core Form I-O
Accounts}\label{visual-analysis-of-a-path-in-core-form-i-o-accounts}

\subsubsection{3.2.1 Build Sets of Nodes and Edges from I-O
Matrices}\label{build-sets-of-nodes-and-edges-from-i-o-matrices}

Build relationship matrices (node-by-node matrix for a given
relationship)

\begin{enumerate}
\def\labelenumi{\alph{enumi})}
\item
  Nodes are (1) Industries, (2) Transactions Nodes have properties (1)
  Subject or (2) Object
\item
  Edges (relationships or Predicates) consist of (1) ``Interacts With''
\item
  Semantic Triple EXAMPLE 1: Subject(Industry 1) \textgreater{}
  Predicate(``Interacts with'') \textgreater{} Object(Transaction 1)
\item
  Semantic Triple EXAMPLE 2: Subject(Transaction 1) \textgreater{}
  Predicate(``Interacts with'') \textgreater{} Object(Industry 2)
\item
  Semantic Chain EXAMPLE 3 where Subject \textbf{\emph{MATCHES}} Object:
  Subject(Industry 1) \textgreater{} Predicate(``Interacts with'')
  \textgreater{} Object(Transaction 1) Subject(Transaction 1)
  \textgreater{} Predicate(``Interacts with'') \textgreater{}
  Object(Industry 2)
\item
  SEMANTIC TRANSACTION CHAIN: (Industry 1) \textgreater{} (``Interacts
  with'') \textgreater{} (Transaction 1) \textgreater{} (``Interacts
  with'') \textgreater{} (Industry 2)
\end{enumerate}

\subsubsection{3.2.2 Visualization of a Path in a
Graph}\label{visualization-of-a-path-in-a-graph}

TBD

\section{4. Leontief Form I-O
Accounts:}\label{leontief-form-i-o-accounts}

\emph{Leontief-Form} I-O accounts map multi-sector inputs to
multi-sector output. This is achieved by deriving a multi-sector
transformation matrix to transform the \emph{Core Form} I-O accounts
into the \emph{Leontief-Form} I-O accounts. The transformation from
\emph{Core Form} to \emph{Netput Form} to \emph{Leontief Form} I-O
accounts (including the derivation of a ``multiplier matrix'') is
derived algebraically as follows:

\[
\begin{align}
    X+f=x && \text{(Core Form I-O Accounts) (6)}\\
    X=_{i}Ax && \text{(Substitute) (7)}\\
    _{i}Ax+f=x && \text{(Core Form I-O Accounts) (8)}\\
        f=(I-_{i}A)x && \text{(Rearrange; Netput Form of I-O Accounts) (9)}\\
    N = (I-_{i}A)^{-1} && \text{(Substitute) (10)}\\
        f=Nx && \text{(Netput Form of I-O Accounts) (11)}\\
    x = N^{-1}f && \text{(Rearrange; Leontief Form I-O Accounts) (12)}\\
    Z = N^{-1} && \text{(Substitute) (13)}\\
    x = Zf && \text{(Leontief Form I-O Accounts) (14)}\\
\end{align}
\]

\subsection{4.1 Leontief Form, Output Space
(Primal)}\label{leontief-form-output-space-primal}

Leontief Form Output Space I-O Accounting identites:

\[
\begin{align}
    x = Zf && \text{(Leontief Form for Total Output) (15)}\\
\end{align}
\]

Total Output vector \([x]_{1\times 6}\):

\[\mathbf{[x]_{1\times 6}} = \left[\begin{array}
{rrr}
x_{1} \\
x_{2} \\
x_{3} \\
x_{4} \\
x_{5} \\
x_{6} \\
\end{array}\right]
\]

Total Final Demand Vector \([f]_{1\times 6}\):

\[\mathbf{[f]_{1\times 6}} = \left[\begin{array}
{rrr}
f_{1} \\
f_{2} \\
f_{3} \\
f_{4} \\
f_{5} \\
f_{6} \\
\end{array}\right]
\]

Leontief Multiplier Matrix \([Z]_{6\times 6}\):

\[\mathbf{[Z]_{6\times 6}} = \left[\begin{array}
{rrr}
z_{11} & z_{12} & z_{13} & z_{14} & z_{15} & z_{16} \\
z_{21} & z_{22} & z_{23} & z_{24} & z_{25} & z_{26} \\
z_{31} & z_{32} & z_{33} & z_{34} & z_{35} & z_{36} \\
z_{41} & z_{42} & z_{43} & z_{44} & z_{45} & z_{46} \\
z_{51} & z_{52} & z_{53} & z_{54} & z_{55} & z_{56} \\
z_{61} & z_{62} & z_{63} & z_{64} & z_{65} & z_{66} \\
\end{array}\right]
\]

\subsection{4.2 Leontief Form, Input Space
(Dual)}\label{leontief-form-input-space-dual}

Leontief Form I-O Input Space Accounting Identites:

\[
\begin{align}
    v = \widehat{_{v}A}Zf && \text{(Leontief Form for Total Value Added) (16)}\\
    m = \widehat{_{m}A}Zf && \text{(Leontief Form for Total Imports) (17)}\\
    (v+m) = (\widehat{_{v}A}Zf+\widehat{_{m}A}Zf) && \text{(Leontief Form for Total Primary Inputs (v+m)) (18)}\\
\end{align}
\]

Total Value Added vector \([v]_{1\times 6}\):

\[\mathbf{[v]_{1\times 6}} = \left[\begin{array}
{rrr}
v_{1} \\
v_{2} \\
v_{3} \\
v_{4} \\
v_{5} \\
v_{6} \\
\end{array}\right]
\]

Total Value Added Direct Coefficients matrix
\([\widehat{_{v}A}]_{6\times 6}\):

\[\mathbf {[\widehat{_{v}A}]_{6\times 6}} = \left[\begin{array}
{rrr}
_{v}a_{11} & 0 & 0 & 0 & 0 & 0 \\
0 & _{v}a_{22} & 0 & 0 & 0 & 0 \\
0 & 0 & _{v}a_{33} & 0 & 0 & 0 \\
0 & 0 & 0 & _{v}a_{44} & 0 & 0 \\
0 & 0 & 0 & 0 & _{v}a_{55} & 0 \\
0 & 0 & 0 & 0 & 0 & _{v}a_{66} \\
\end{array}\right]
\]

Total Final Demand Vector \([f]_{1\times 6}\):

\[\mathbf{[f]_{1\times 6}} = \left[\begin{array}
{rrr}
f_{1} \\
f_{2} \\
f_{3} \\
f_{4} \\
f_{5} \\
f_{6} \\
\end{array}\right]
\]

Leontief Form Multiplier Matrix \([Z]_{6\times 6}\):

\[\mathbf{[Z]_{6\times 6}} = \left[\begin{array}
{rrr}
z_{11} & z_{12} & z_{13} & z_{14} & z_{15} & z_{16} \\
z_{21} & z_{22} & z_{23} & z_{24} & z_{25} & z_{26} \\
z_{31} & z_{32} & z_{33} & z_{34} & z_{35} & z_{36} \\
z_{41} & z_{42} & z_{43} & z_{44} & z_{45} & z_{46} \\
z_{51} & z_{52} & z_{53} & z_{54} & z_{55} & z_{56} \\
z_{61} & z_{62} & z_{63} & z_{64} & z_{65} & z_{66} \\
\end{array}\right]
\]

Intermediate product matrix \([\widehat{_{v}A}Z]_{6\times 6}\):

\[\mathbf{[\widehat{_{v}A}Z]_{6\times 6}} = \left[\begin{array}
{rrr}
(_{v}a_{11}*z_{11}) & (_{v}a_{11}*z_{12}) & (_{v}a_{11}*z_{13}) & (_{v}a_{11}*z_{14}) & (_{v}a_{11}*z_{15}) & (_{v}a_{11}*z_{16}) \\
(_{v}a_{21}*z_{21}) & (_{v}a_{21}*z_{22}) & (_{v}a_{21}*z_{23}) & (_{v}a_{21}*z_{24}) & (_{v}a_{21}*z_{25}) & (_{v}a_{21}*z_{26}) \\
(_{v}a_{31}*z_{31}) & (_{v}a_{31}*z_{32}) & (_{v}a_{31}*z_{33}) & (_{v}a_{31}*z_{34}) & (_{v}a_{31}*z_{35}) & (_{v}a_{31}*z_{36}) \\
(_{v}a_{41}*z_{41}) & (_{v}a_{41}*z_{42}) & (_{v}a_{41}*z_{43}) & (_{v}a_{41}*z_{44}) & (_{v}a_{41}*z_{45}) & (_{v}a_{41}*z_{46}) \\
(_{v}a_{51}*z_{51}) & (_{v}a_{51}*z_{52}) & (_{v}a_{51}*z_{53}) & (_{v}a_{51}*z_{54}) & (_{v}a_{51}*z_{55}) & (_{v}a_{51}*z_{56}) \\
(_{v}a_{51}*z_{61}) & (_{v}a_{51}*z_{62}) & (_{v}a_{51}*z_{63}) & (_{v}a_{51}*z_{64}) & (_{v}a_{51}*z_{65}) & (_{v}a_{51}*z_{66}) \\
\end{array}\right]
\]

Total Value Added vector v in explicit computational form
\([v]_{1\times 6}=[[\widehat{_{v}A}Z][f]]_{1\times 6}\):

\[\mathbf{[v]_{1\times 6}=[[\widehat{_{v}A}Z][f]]_{1\times 6}} = \left[\begin{array}
{rrr}
((_{v}a_{11}*z_{11})*f_{1}) & +((_{v}a_{11}*z_{12})*f_{2}) & +((_{v}a_{11}*z_{13})*f_{3}) & +((_{v}a_{11}*z_{14})*f_{4}) & +((_{v}a_{11}*z_{15})*f_{5}) & +((_{v}a_{11}*z_{16})*f_{6}) \\
((_{v}a_{21}*z_{21})*f_{1}) & +((_{v}a_{21}*z_{22})*f_{2}) & +((_{v}a_{21}*z_{23})*f_{3}) & +((_{v}a_{21}*z_{24})*f_{4}) & +((_{v}a_{21}*z_{25})*f_{5}) & +((_{v}a_{21}*z_{26})*f_{6}) \\
((_{v}a_{31}*z_{31})*f_{1}) & +((_{v}a_{31}*z_{32})*f_{2}) & +((_{v}a_{31}*z_{33})*f_{3}) & +((_{v}a_{31}*z_{34})*f_{4}) & +((_{v}a_{31}*z_{35})*f_{5}) & +((_{v}a_{31}*z_{36})*f_{6}) \\
((_{v}a_{41}*z_{41})*f_{1}) & +((_{v}a_{41}*z_{42})*f_{2}) & +((_{v}a_{41}*z_{43})*f_{3}) & +((_{v}a_{41}*z_{44})*f_{4}) & +((_{v}a_{41}*z_{45})*f_{5}) & +((_{v}a_{41}*z_{46})*f_{6}) \\
((_{v}a_{51}*z_{51})*f_{1}) & +((_{v}a_{51}*z_{52})*f_{2}) & +((_{v}a_{51}*z_{53})*f_{3}) & +((_{v}a_{51}*z_{54})*f_{4}) & +((_{v}a_{51}*z_{55})*f_{5}) & +((_{v}a_{51}*z_{56})*f_{6}) \\
((_{v}a_{51}*z_{61})*f_{1}) & +((_{v}a_{51}*z_{62})*f_{2}) & +((_{v}a_{51}*z_{63})*f_{3}) & +((_{v}a_{51}*z_{64})*f_{4}) & +((_{v}a_{51}*z_{65})*f_{5}) & +((_{v}a_{51}*z_{66})*f_{6}) \\
\end{array}\right]
\]

\subsection{4.3 Column space expansion of Leontief Form I-O Output
Space:}\label{column-space-expansion-of-leontief-form-i-o-output-space}

Equation:

\[
\begin{align}
    x = Zf && \text{(Leontief Form for Total Output) (15)}\\
    O = Z\hat{f} && \text{($\hat{f}$ is a diagonal matrix of final demands; O is the column space of x)}\\
\end{align}\]

Where: Vector \([\hat{f}]_{1\times 6}\):

\[\mathbf {[\hat{f}]_{1\times 6}} = \left[\begin{array}
{rrr}
x_{1} \\
x_{2} \\
x_{3} \\
x_{4} \\
x_{5} \\
x_{6} \\
\end{array}\right]
\]

Vector f:

\[\mathbf{f} = \left[\begin{array}
{rrr}
f_{11} & 0 & 0 & 0 & 0 & 0 \\
0 & f_{22} & 0 & 0 & 0 & 0 \\
0 & 0 & f_{33} & 0 & 0 & 0 \\
0 & 0 & 0 & f_{44} & 0 & 0 \\
0 & 0 & 0 & 0 & f_{55} & 0 \\
0 & 0 & 0 & 0 & 0 & f_{66} \\
\end{array}\right]
\]

Matrix Z:

\[\mathbf{Z} = \left[\begin{array}
{rrr}
z_{11} & z_{12} & z_{13} & z_{14} & z_{15} & z_{16} \\
z_{21} & z_{22} & z_{23} & z_{24} & z_{25} & z_{26} \\
z_{31} & z_{32} & z_{33} & z_{34} & z_{35} & z_{36} \\
z_{41} & z_{42} & z_{43} & z_{44} & z_{45} & z_{46} \\
z_{51} & z_{52} & z_{53} & z_{54} & z_{55} & z_{56} \\
z_{61} & z_{62} & z_{63} & z_{64} & z_{65} & z_{66} \\
\end{array}\right]
\]

Column-space expansion Matrix O:

\[\mathbf{O} = \left[\begin{array}
{rrr}
o_{11} & o_{12} & o_{13} & o_{14} & o_{15} & o_{16} \\
o_{21} & o_{22} & o_{23} & o_{24} & o_{25} & o_{26} \\
o_{31} & o_{32} & o_{33} & o_{34} & o_{35} & o_{36} \\
o_{41} & o_{42} & o_{43} & o_{44} & o_{45} & o_{46} \\
o_{51} & o_{52} & o_{53} & o_{54} & o_{55} & o_{56} \\
o_{61} & o_{62} & o_{63} & o_{64} & o_{65} & o_{66} \\
\end{array}\right]
\]

\subsection{4.5 Visualization of Tier 1 Column Space
Expansion}\label{visualization-of-tier-1-column-space-expansion}

TBD

\subsection{4.6 Column space expansion of Leontief Form I-O Input
Space:}\label{column-space-expansion-of-leontief-form-i-o-input-space}

TBD

\subsection{4.7 Visualization of Tier 1 Column Space
Expansion}\label{visualization-of-tier-1-column-space-expansion-1}

\section{5. Column Space Expansion of
RPVCs}\label{column-space-expansion-of-rpvcs}

\subsection{\texorpdfstring{5.1 Supply Chain Tier 1 Column Space
Expansion of Total Output of Vector
o\textsubscript{5}}{5.1 Supply Chain Tier 1 Column Space Expansion of Total Output of Vector o5}}\label{supply-chain-tier-1-column-space-expansion-of-total-output-of-vector-o5}

\textbf{\emph{SECTION 3.3 TBD!!!!}}

Column space expansion of output vector o\textsubscript{5}:

\[
\begin{align}
    o = Zf && \text{(Leontief transformation function)}\\
    O = Z\hat{f} && \text{(f-hat is a diagonal matrix of intermediate demands by Biopwr; O is the column space of o)}\\
\end{align}
\]

(Product Electricity from Biomass Power Generation) Vector
o\textsubscript{5}:

\[\mathbf{o_5} = \left[\begin{array}
{rrr}
o_{15} \\
o_{25} \\
o_{35} \\
o_{45} \\
o_{55} \\
o_{65} \\
\end{array}\right]
\]

Column space expansion of output vector o\textsubscript{5}:

\[
\begin{align}
    o = Zf && \text{(Leontief transformation function)}\\
    O = Z\hat{f} && \text{(f-hat is a diagonal matrix of intermediate demands by Biopwr; O is the column space of o)}\\
\end{align}
\]

Matrix O:

\[\mathbf{O} = \left[\begin{array}
{rrr}
o_{11} & o_{12} & o_{13} & o_{14} & o_{15} & o_{16} \\
o_{21} & o_{22} & o_{23} & o_{24} & o_{25} & o_{26} \\
o_{31} & o_{32} & o_{33} & o_{34} & o_{35} & o_{36} \\
o_{41} & o_{42} & o_{43} & o_{44} & o_{45} & o_{46} \\
o_{51} & o_{52} & o_{53} & o_{54} & o_{55} & o_{56} \\
o_{61} & o_{62} & o_{63} & o_{64} & o_{65} & o_{66} \\
\end{array}\right]
\]

\subsection{\texorpdfstring{5.2 Supply Chain Tier 2 Column Space
Expansion of Total Output of Vector
o\textsubscript{4}}{5.2 Supply Chain Tier 2 Column Space Expansion of Total Output of Vector o4}}\label{supply-chain-tier-2-column-space-expansion-of-total-output-of-vector-o4}

\subsection{\texorpdfstring{5.3 Supply Chain Tier 3 Column Space
Expansion of Total Output of Vector
o\textsubscript{3}}{5.3 Supply Chain Tier 3 Column Space Expansion of Total Output of Vector o3}}\label{supply-chain-tier-3-column-space-expansion-of-total-output-of-vector-o3}

\subsection{\texorpdfstring{5.4 Supply Chain Tier 4 Column Space
Expansion of Total Output of Vector
o\textsubscript{2}}{5.4 Supply Chain Tier 4 Column Space Expansion of Total Output of Vector o2}}\label{supply-chain-tier-4-column-space-expansion-of-total-output-of-vector-o2}

\section{6. Supply Chain
Visualization}\label{supply-chain-visualization}

\subsection{6.1 Assemble a Supply Chain
Knowlegebase}\label{assemble-a-supply-chain-knowlegebase}

\subsection{6.2 Query Knowlegebase for Supply Chain
Pattern}\label{query-knowlegebase-for-supply-chain-pattern}

\subsection{6.3 Visualization of Supply
Chain}\label{visualization-of-supply-chain}

\[\begin{align}
   \begin{bmatrix}
           W_{11} \\
           W_{12} \\
           \vdots \\
           W_{1n}
   \end{bmatrix}
   &+ \begin{bmatrix}
           W_{21} \\
           W_{22} \\
           \vdots \\
           W_{2n}
       \end{bmatrix}
   &+ \dots
   &+ \begin{bmatrix}
           W_{n1} \\
           W_{n2} \\
           \vdots \\
           W_{nn}
       \end{bmatrix}
   &= \begin{bmatrix}
           \frac{W_{11} + W_{21} + \dots + W_{n1}}{n} \\
           \frac{W_{12} + W_{22} + \dots + W_{n2}}{n} \\
           \vdots \\
           \frac{W_{1n} + W_{2n} + \dots + W_{nn}}{n}
    \end{bmatrix}
\end{align}
\]

```




\end{document}
